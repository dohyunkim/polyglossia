% !TeX TS-program = xelatex
\documentclass[a4paper]{book}
\usepackage{polyglossia}
\usepackage{graphicx}
\usepackage{url,amsmath}

\setmainlanguage{uyghur}
\newfontfamily\arabicfont[Script=Arabic,Scale=1]{UKIJ Tuz}
\let\arabicfonttt\ttfamily

\setotherlanguage[variant=british]{english}

\title{لاتېكىس قوللانمىسى}
\author{neouyghur}
\date{\today}

\begin{document}

\maketitle

\tableofcontents

\listoffigures

\listoftables

\pagenumbering{arabic}

\chapter{ئۇيغۇرلار}

\section{ئۇيغۇرلار ۋە ئۇلارنىڭ تۇرمۇش مەدەنىيىتى}

ئۇيغۇرلار شىنجاڭ ئۇيغۇر ئاپتونوم رايونىنىڭ ھەقىقىي ئىگىلىرى ۋە مىراسخورلىرى بولۇپ، بۇ رايوندا توپلىشىپ ياشايدىغان غوللۇق، يەرلىك مىللەت ۋە ئاپتونومىيە ھوقۇقى يۈرگۈزگۈچى مىللەت. ئۇيغۇرلار پارچە ھالەتتە، جۇڭگونىڭ بىر قاتار چوڭ شەھەرلىرىدە خىزمەت ۋە تىجارەت قىلىدۇ. ئۇيغۇرلار شىنجاڭ ئۇيغۇر ئاپتونوم رايونىدىن باشقا، يەنە جۇڭگو، قازاقىستان، قىرغىزىستان، ئۆزبېكىستان، تاجىكىستان، تۈركمەنىستان (ئاساسلىقى بايرام ئەلى رايونى)، پاكىستان، ئەرەبىستان، تۈركىيە، ئاۋسترالىيە، گېرمانىيە، نورۋىگىيە، گوللاندىيە، شىۋىتسىيە، فىنلاندىيە، بېلگىيە، روسيە، ئامېرىكا قاتارلىق دۆلەتلەردە كۆرۈنەرلىك نوپۇستا، مۇئەييەن جامائەت تۈركۈمى شەكلىدە ياشايدۇ.

\begin{english}
The \textbf{Uyghurs}\footnote{%
	From \url{https://en.wikipedia.org/wiki/Uyghurs}}  have traditionally inhabited a series of oases scattered across the Taklamakan Desert within the Tarim Basin. These oases have historically existed as independent states or were controlled by many civilizations including China, the Mongols, the Tibetans and various Turkic polities. The Uyghurs gradually started to become Islamized in the 10th century and most Uyghurs identified as Muslims by the 16th century. Islam has since played an important role in Uyghur culture and identity. 
\end{english}

\section{نىكاھ}
ئۇيغۇرلاردا نىكاھ مۇناسىۋىتى بىلەن چاي ئىچكۈزۈش، توي ئالدىدىكى مەسلىھەت چايلىرى، نىكاھ ئوقۇپ قىز-يىگىتنىڭ رازىلىقىنى ئېلىش، توي مۇراسىمى، يىگىت-قىزلار ئولتۇرۇشى، قۇدىلار چىللاقلىرى ئىزچىل ساقلىنىپ كەلدى. ئۇيغۇرلاردا دەپنە مۇراسىمى ھەرقايسى دىنىي مەدەنىيەتلەر تەسىرىدە ھەرخىل بولسىمۇ، ئەمما مېيىتنى پاكىزە يۇيۇپ كېپەنلەش، ھازىدارلار ئاق رومال سېلىپ، ئاق بەلۋاغ باغلاپ يىغا-زارە قىلىش، مېيىتنىڭ نامىزىنى چۈشۈرۈش، جىنازىنى ئالمىشىپ تالىشىپ كۈتۈرۈپ قەبرىستانلىققا ئېلىپ بېرىش، لەھەتتە مېيىتنىڭ يۈزىنى قىبلە (غەرب) تەرەپكە قىلىپ ياتقۇزۇش، ئىچ گۆرنىڭ ئاغزىنى كېسەك بىلەن ئېتىپ، تاش گۆرنى توپا بىلەن كۆمۈش، قەبرە بېشىدا مېيىتنىڭ ئىجابىي تەرىپىگە گۇۋاھلىق بېرىش، قەبرە تېشى-گۈمبەز ئورنىتىش، مېيىتنىڭ يەتتە، قىرىق، يىل نەزىرلىرىنى ئۆتكۈزۈش ئادەتلىرى بىردەك ئىزچىل بۇلۇپ كەلدى. ئۇيغۇرلار ۋە ئۇلارنىڭ ئەجدادلىرى مېيىت سۆڭىكىنى كاھىش (ساپال) ساندۇققا سېلىپ يەرلىككە قۇيۇش، مېيىتنى تاش گۆرگە كۆمۈش، مېيىتنى ئاستىغا ياغاچ شادا قويۇلغان گۆرگە ياتقۇزۇپ، ئۈستىگە قىزىل تۇپراق ۋە قۇم تۆكۈپ كۆمۈش، شام گۆرگە قويۇش قاتارلىق دەپنە قىلىش ئۇسۇللىرىنىمۇ قوللاندى. 

\section{ھېيت-ئايەم}


\begin{figure}
	\begin{center}
\includegraphics[scale=1]{example-image-a}
\caption{ئا رەسىم.}
	\end{center}
\end{figure}

\begin{table}
\begin{center}
	\begin{tabular}{||c c c c||} 
		\hline
		Col1 & Col2 & Col2 & Col3 \\ [0.5ex] 
		\hline\hline
		1 & 6 & 87837 & 787 \\ 
		\hline
		2 & 7 & 78 & 5415 \\
		\hline
		3 & 545 & 778 & 7507 \\
		\hline
		4 & 545 & 18744 & 7560 \\
		\hline
		5 & 88 & 788 & 6344 \\ [1ex] 
		\hline
	\end{tabular}
	\caption{بىر سىناق جەدۋەل}
\end{center}
\end{table}

\begin{equation}
	x^2 + y^2 = z^2
	\label{test}
\end{equation}


\chapter{مەڭگۈ تاش يادىكارلىقلىرى}

\section{تەس مەڭگۈ تېشى}
بۇ مەڭگۈ تاش يادىكارلىقى 1976- يىلى موڭغۇلىيەدىكى تەس دەرياسىنىڭ سول قىرغىدىكى نوغۇن تولغوي ئىگىزلىكىنىڭ يېنىدىن تېپىلغانلىقى ئۈچۈن، «تەس مەڭگۈ تېشى» دەپ ئاتالغان. «تەس مەڭگۈ تېشى» ئورخۇن ئۇيغۇر خانلىقى دەۋرىگە مەنسۇپ يادىكارلىقلارنىڭ ئىچىدە بۇزغۇنچىلىققا ئۇچرىشى ئېغىرراق. ئەڭ مۇھىم بولغان يازما يادىكارلىق. ئۇنىڭدا بۆگۈ قاغان (759- 780- يىللار) ئەجدادى (دادىسى) بولغان ئەل ئەتمىش بىلگە قاغان (مويۇنچۇر، 747- 759- يىللار9 دەۋرىگىچە بولغان ئۇيغۇر قاغانلىرىنىڭ تارىخى، جۈملىدىن بىرىنچى ۋە ئىككىنچى ئۇيغۇر خانلىقىنىڭ تەقدىرى. كۆل بىلگە قاغان ۋە ئەل ئەتمىش بىلگە قاغان باشچىلىقىدىكى ئۈچىنچى ئۇيغۇر خانلىقىنىڭ بەرپا بولۇشى قاتارلىق مەسىلىلەر ھەققىدە مەلۇمات بېرىلگەن. بۇ ھەقتىكى بىر قىسىم مەزمۇنلار كونا- يېڭى تاڭنامىلەردە ئۇچرىمايدۇ. شۇڭا تەس مەڭگۈ تېشىدىكى مەلۇماتلار زور سېلىشتۇرما قىممەتكە ئىگە. تەتقىقاتلارغا قارىغاندا، «تەس مەڭگۈ تېشى» نىڭ ئاپتورى مويۇنچۇر قاغاننىڭ يېقىن تۇققىنى تۈپەك ئالىپ شۇل دېگەن كىشى بولۇپ، مەڭگۈ تاشنى قاغان يېڭىدىن تەختكە چىققان ۋاقىتتا توغرىسى 761- 762- يىللىرى ئورنىتىلغان، دېيىشكە بولىدۇ.


\begin{table}
	\begin{center}
		\begin{tabular}{||c c c c||} 
			\hline
			Col1 & Col2 & Col2 & Col3 \\ [0.5ex] 
			\hline\hline
			1 & 6 & 87837 & 787 \\ 
			\hline
			2 & 7 & 78 & 5415 \\
			\hline
			3 & 545 & 778 & 7507 \\
			\hline
			4 & 545 & 18744 & 7560 \\
			\hline
			5 & 88 & 788 & 6344 \\ [1ex] 
			\hline
		\end{tabular}
		\caption{يەنە بىر سىناق جەدۋەل}
	\end{center}
\end{table}

\section{تېرخىن مەڭگۈ تېشى}

\begin{figure}
	\begin{center}
	\includegraphics[scale=0.5]{example-image-b}
	\caption{ب رەسىم.}
	\end{center}
\end{figure}

\begin{thebibliography}{9}
	\bibitem{latexcompanion} 
	Michel Goossens, Frank Mittelbach, and Alexander Samarin. 
	\textit{The \LaTeX\ Companion}. 
	Addison-Wesley, Reading, Massachusetts, 1993.
	
	\bibitem{einstein} 
	ئو. تۇرسۇن. 
	\textit{لاتېكىست قوللانمىسى}. (ئۇيغۇرچە) 
	[\textit{بىر ژورنال}]. , 322(10):891–921, 1905.
\end{thebibliography}

\end{document}
